% RESULTADOS-------------------------------------------------------------------

\chapter{APRESENTAÇÃO E ANÁLISE DE RESULTADOS}

Aqui será detalhado os resultados obtidos com o funcionamento do circuito desenvolvido integrado à uma forno microondas convencional da marca Panasonic. Nesta capítulo será descrito o funcionamento prático do circuito, exposto o resultado da medições de parâmetros que avaliam a performance e feitas comparações com os parâmetros do circuito ferrorressonante, disponíveis na literatura. 


\section{Sistema em Funcionamento}
A placa original do aparelho foi removida e trocada pela placa desenvolvida, ligando-se todos os fios que conectam o circuito projetado à rede e ao magnetron maneira similar à original. Para acionar o circuito, foi utilizada a interface já presente no forno, configurando o nível percentual de potência e o tempo de cozimento de acordo com os experimentos realizados. Para testar o funcionamento, a equipe acionou o circuito em diferentes configurações de potência, medindo-se as formas de onda em diferentes pontos para demonstrar o funcionamento prático do sistema. As subseções abaixo irão descrever o funcionamento prático do sistema

\subsection{Tensão, corrente e potência}

\subsection{Chaveamento dos IGBTs}

\subsection{Sinais de controle e status}


\section{Medições de parâmetros}
Para comparar o desempenho do circuito desenvolvido com uma fonte ferrorressonante tradicional, foram medidos três parâmetros essenciais para se possibilitar uma discussão válida sobre o tema: eficiência, fator de potência e dimensões físicas, incluindo massa, altura, largura e comprimento. 

\section{Eficiência}
A eficiência do forno microondas pode ser avaliada de forma geral como a performance do circuito de alimentação do magnetron. Este parâmetro pode ser obtido fazendo-se uma relação entre o calor absorvido pelo líquido dentro na câmara de cozimento do forno e a energia elétrica consumida pelo circuito.
Para avaliar a performance, a equipe repetiu um experimento que é utilizado pelo INMETRO para determinar a eficiência de fornos microondas, o qual está detalhado em \cite{Inmetro}. O experimento consiste em esquentar 1 litro de água em um determinado tempo. A água utilizada para este fim estava armazenada em um recipiente aberto de vidro, pesando 400(?)g. Antes da realização da experiência, a equipe aqueceu o recipiente vazio para determinar se o mesmo aquecia de forma significativa sem a presença de água, a fim de verificar se o recipiente poderia interferir nas medições. Inicialmente o recipiente se encontrava a X °C. Após colocá-lo no microondas por 60 (?) segundos, o recipiente apresentou temperatura de X + dx ºC. Logo a variação foi de dx ºC, a qual não acarreta em impactos significativos na precisão das medidas.
Inicialmente mediu-se a temperatura da água no recipiente, utilizando-se um termopar industrial comum. Feito isso, o recipiente foi colocado no forno e acionado em potência 80\% durante 60(?) segundos. Então, retirou-se a o recipiente do aparelho e com o termopar mediu-se a temperatura. Através das fórmulas descritas em cite, mediu-se a eficiência do circuito. No total, foram realizados dois experimentos, cada qual possuindo uma temperatura da água distinta. As figuras abaixo mostra a imagem das leituras do termopar antes e depois de cada experimento, enquanto a tabela z resume os resultados obtidos com as leituras
(figgies go here)
(tablies go there)


Da tabela z, percebe-se que o valor da eficiência obtida pelos experimentos é muito próximo. Isto demonstra que o circuito possui um comportamento regular, apresentando pouca variação conforme o aumento de temperatura da água. Este comportamento é relevante pois mostra que o calor transmitido mantém-se constante em determinada faixa de temperatura, sugerindo que a uniformidade do aquecimento tende a se manter a mesma ao passo que a temperatura aumenta. 
