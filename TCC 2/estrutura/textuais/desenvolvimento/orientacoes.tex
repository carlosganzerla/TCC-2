% ORIENTAÇÕES GERAIS------------------------------------------------------------
\chapter{CONSIDERAÇÕES FINAIS}
\label{chap:consideracoesFinais}

Com todas as etapas deste trabalho tendo sido realizadas, podem-se tomar várias conclusões sobre os resultados obtidos, sobre os softwares criados, sobre os equipamentos utilizados, sobre as dificuldades vividas durante o desenvolvimento do projeto e, também, principalmente, sobre o aprendizado adquirido por todos os membros da equipe ao realizar cada etapa deste projeto.

No desenvolvimento deste trabalho, foram utilizados conhecimentos de engenharia de software, de modelamento de banco de dados e de desenvolvimento de websites. Foram utilizado na prática conhecimentos de visão computacional e implementados algoritmos de detecção e identificação de pessoas numa filmagem. Também foram utilizados conhecimentos de eletrônica digital para a confecção de um circuito de comunicação serial, além da utilização de um sistema de telecomunicações. 

Apesar dessa possibilidade não ter sido estressada neste protótipo, o sistema de segurança aqui abordado possibilita a integração de várias câmeras simultâneamente para vários usuários diferentes. Tal possibilidade abre um leque enorme de possibilidades de continuação para um projeto que solucionaria problemas de uma parcela considerável da população brasileira. O baixo custo dos materiais possibilita a implementação do sistema nos mais variados ambientes. De forma geral, pode-se concluir que a ideia deste tipo de sistema traria benefícios à sociedade brasileira que, nos dias de hoje, infelizmente, ainda sente falta de segurança dentro de seu próprio lar.