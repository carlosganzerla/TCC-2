% METODOLOGIA------------------------------------------------------------------
\chapter{PROCEDIMENTOS METODOLÓGICOS}
\label{chap:metodologia}

Este capítulo tem como finalidade descrever a metodologia e os procedimentos adotados na confecção deste projeto, bem como também realizar uma consolidação de todos os métodos aqui utilizados e apresentar o funcionamento do sistema como um todo. Para tal, os procedimentos metodológicos foram divididos da seguinte forma: Fonte inversora (Seção 3.1), Fonte de alimentação (Seção 3.2),  Realimentação de corrente (Seção 3.3), DSPIC33  (Seção 3.4), Circuito de controle (Seção 3.5), Integração dos componentes (Seção 3.6)

\section{FONTE INVERSORA}
\label{sec:fonteInversora}

Para realizar a alimentação do magnetron, foi utlizada uma fonte inversora, a qual consiste em um inversor ressonanete classe E. Segundo \citeonline{Hidenori1991}, uma fonte inversora tem as seguintes vantagens em relação à uma fonte ferrorressonante tradicional:
\begin{itemize}
    \item Potência de saída controlável;
    \item Maior eficiência energética;
    \item Circuito menor e mais leve;
    \item Pode operar em maior frequência.
\end{itemize} 

\begin{figure}[!htb]
    \centering
    \includegraphics[width=0.9\textwidth]{./dados/figuras/font_inverter}
    \caption{Fonte inversora para alimentação do magnetron}
    \fonte{\citeonline{Hidenori1991}}
    \label{fig:figura-inverter}
\end{figure}

\subsection{Simulações}
\label{sec:simulations}

Para verificar se a fonte inversora é viável para o projeto, foram feitas simulações do circuito no software \textit{PSIM}. Na alimentação o magnetron, são necessários cerca de 4 kV. Assim, primeiramente foi desenvolvido um circuito para simular a alimentação de uma carga de cerca de 100 M$\Omega$, com uma tensão de entrada de 127 V. A figura a seguir mostra o circuito desenvolvido:

\begin{figure}[!htb]
    \centering
    \includegraphics[width=0.9\textwidth]{./dados/figuras/psim1}
    \caption{Circuito da fonte inversora simulada}
    \fonte{Autoria própria (2019)}
    \label{fig:circ_sim_1}
\end{figure}

Para averiguar se uma fonte inversora consegue alimentar uma carga de alta potência à uma tensão de alguns kV, o inversor foi chaveado em um ciclo de trabalho de 50\%. A figura abaixo mostra a froma de onda da tensão na carga resistiva:

\begin{figure}[!htb]
    \centering
    \includegraphics[width=0.9\textwidth]{./dados/figuras/psim2}
    \caption{Forma da onda simulada da tensão na carga}
    \fonte{Autoria própria (2019)}
    \label{fig:figura-graf_sim_1}
\end{figure}

Na figura \ref{fig:figura-graf_sim_1}, pode-se ver que o pico de tensão da carga chega à quase 4 kV, o que já é suficiente para o objetivo em questão. Logo, conclui-se que a fonte inversora é viável para a alimentação do circuito de um magnetron.

\subsection{Projeto}

O projeto da fonte inversora se baseou em boa parte no conceito desenvolvido por \citeonline{Hidenori1991}. O chaveamento do circuito é feito por um microcontrolador que é realimentado por um sensor de corrente. Foram utilizados IGBTs em paralelo para aumentar a potência dos sistemas e reduzir perdas no dispositivo. Um transformador de três fios de alta potência alimenta as entradas que são ligadas ao magnetron com alta tensão. A tensão de entrada do circuito é a tensão da rede retificada por uma ponte de diodos. O circuito foi desenvolvido no software \texit{Altium Desginer}. A figura \ref{fig:proj-font-inv} mostra o circuito projetado:

\begin{figure}[!htb]
    \centering
    \includegraphics[width=1.1\textwidth]{./dados/figuras/proj-font-inv}
    \caption{Circuito projetado}
    \fonte{Autoria própria (2019)}
    \label{fig:proj-font-inv}
\end{figure}

\subsection{Descrição de funcionamento}

\subsection{Montagem}


\section{\texit{FONTE DE ALIMENTAÇÃO}}
\label{sec:font}

\subsection{Montagem}


\section{\texit{REALIMENTAÇÃO DE CORRENTE}}
\label{sec:shunt}
Devido à natureza do circuito, a realimentação por corrente foi escolhida. Para possibilitar esta realimentação, foi desenvolvida uma interface shunt analógica que é ligada diretamente ao ADC do microcontrolador. A interface consiste em um circuito que irá condicionar o sinal da tensão aplicada em um resistor shunt de 30 m $\Omega$ para os pinos de \textit{input} do microcontrolador. A figura abaixo mostra a interface projetada:

\begin{figure}[!htb]
    \centering
    \includegraphics[width=0.9\textwidth]{./dados/figuras/proj-shunt}
    \caption{Condicionador de sinal de realimentação}
    \fonte{Autoria própria (2019)}
    \label{fig:proj-font-inv}
\end{figure}

\subsection{Montagem}

\section{DSPIC33}
\label{sec:dsPIC}

O DSPIC33 é um microntrolador da família PIC, desenvolvido pela Microchip Technology Inc., que possui funcionalidade de processador digital de sinais (DSP). Este componente foi escolhido para fazer o controle do chaveamento da fonte inversora pois consegue operar em uma ampla faixa de temperatura e possui diversas funcionalidades interessantes para o controle de sinais analógicos de alta frequência. Algumas das funcionalidades, cruciais para o projeto, incluem:

\begin{itemize}
    \item Módulo ADC configurável de 10 bits e amostragem de 1.1 Msps ou 12 bits e amostragem de 500 ksps;
    \item Três amplificadores operacionais integrados ao ADC da plataforma;
    \item Interrupções de \textit{Change Notification} em todos os pinos de I/O;
    \item \textit{Timers} e contadores de 32 bits;
    \item Funções de PWM de alta velocidade.
\end{itemize} 

\begin{figure}[!htb]
    \centering
    \includegraphics[width=0.3\textwidth]{./dados/figuras/dspic}
    \caption{Microcontrolador DSPIC33}
    \fonte{Autoria própria (2019)}
    \label{fig:figura-dspic}
\end{figure}

A CPU da plataforma possui arquitetura Harvard, típica da família de processadores PIC, possuindo uma palavra de instrução de 24 bits e 12 MB de endereços de memória de programa. O microprocessador possui um extenso suporte ao processamento digital de sinais, tendo acumuladores e ULA de 40 bits, dois multiplicadores de alta velocidade 17 por 17 bits e um \textit{barrell shifter} de 40 bits que consegue alternar 16 bits em úniclo ciclo de \textit{clock}. A arquitetura do processador fornece uma compilação eficiente de código, suportando a linguagem C e \textit{Assembly}.

No circuito, o microcontrolador recebe o sinal de corrente  condicionado dos terminais da interface ligada ao \textit{shunt} e converte este sinal para um valor discreto. No programa do microprocessador, este valor é utilizado no algorítmo de controle, que processa digitalmente o sinal recebido e faz o chaveamento dos IGBTs.

\begin{figure}[!htb]
    \centering
    \includegraphics[width=0.9\textwidth]{./dados/figuras/proj-uc}
    \caption{Esquemático do microcontrolador no circuito}
    \fonte{Autoria própria (2019)}
    \label{fig:figura-dspic}
\end{figure}


\section{CIRCUITO DE CONTROLE}
\label{sec:controlCircuit}



\section{INTEGRAÇÃO DOS COMPONENTES}
\label{sec:plant}

