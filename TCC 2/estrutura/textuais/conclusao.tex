% CONCLUSÃO------------------------------------------------------------
\chapter{CONCLUSÃO}
\label{chap:conclusao}


No desenvolvimento deste trabalho, de forma geral foram utilizados conhecimentos de engenharia de \textit{software}, de modelamento de banco de dados e de desenvolvimento de \textit{websites}. Foram utilizados na prática conhecimentos de visão computacional e implementados algoritmos de detecção e identificação de pessoas numa filmagem. Também foram utilizados conhecimentos de eletrônica digital para a confecção de um circuito de comunicação serial, além da utilização de um sistema de telecomunicações. Os objetivos citados na introdução foram todos alcançados, alguns com algumas limitações não previstas quando o projeto foi idealizado.

Começando pela ordem dos objetivos citados na introdução: a configuração de uma cerca virtual foi atingida, permitindo ao usuário criar uma área dentro da imagem da câmera e dessa forma só gerar eventos se esse perímetro for ultrapassado. A interface via \textit{web} para o usuário permite a criação de uma área em qualquer formato, dessa forma adequa-se a necessidade do usuário.
Para a detecção de eventos e a identificação da causa foi necessário realizar alguns ajustes devido ao \textit{hardware} escolhido. A limitação mais importante foi em relação a quantidade de \textit{frames} processados por segundo que foi obtido nos nossos testes. Iniciamos com o objetivo de 10 \textit{FPS}, mas devido o alto processamento que a rede neural requer para a identificação diminuímos para 5 \textit{FPS}, dessa forma o processamento conseguiu continuar em tempo real (considerando a resolução de 640x480 da câmera).

Pode-se concluir que a distância da câmera influencia na acertividade do reconhecimento. No entanto, não é o único parâmetro e por isso relacionar apenas esses dois dados em um gráfico não seria uma análise justa do ponto de vista de desempenho do projeto, tendo em vista que outros parâmetros observados também interferem na acertividade, como, por exemplo, a luminosidade do ambiente (por causa da subtração de fundo) e a posição da pessoa (frontal, lateral ou de costas).

O envio do alerta também foi concluído, o usuário na tela de configurações permite adicionar um número de celular e ligar as notificações individuais para cada câmera.

Outro objetivo específico era a integração do \textit{appliance} com o servidor e o armazenamento dos vídeos, ambos foram concluídos. O vídeo gerado do evento é armazenado na placa localmente e somente se o usuário requisitar \textit{online} é transferido para o servidor usando o protocolo \textit{SCP}. Assim como é possível acessar a placa remotamente atraves do servidor web via protocolo \textit{SSH}.

Em termos de \textit{cybersecurity}, além da utilização do protocolo \textit{SSH} para a transferência dos vídeos e imagens, é feita uma verificação de acesso dos usuários, averiguando se os mesmos possuem permissão para visualizar o evento em questão. Foi considerada a possibilidade de se criptografar esses dados. No entanto, por causa de um possível problema de processamento na \textit{Raspberry}, que seria causado por processar os movimentos e realizar a criptografia das imagens ao mesmo tempo, foi decidido que esta medida de segurança não seria desenvolvida. No caso deste projeto tornar-se um produto de fato, é de grande importância que esse tema seja revisto.

O dimensionamento de \textit{hardware} previsto inicialmente mostrou suas limitações, forçando a diminuição de alguns dos requisitos previstos, como a taxa de \textit{FPS}, por exemplo. Pensando em uma aplicação prática - como monitoramento residencial - é possível aumentar a quantidade de objetos a serem reconhecidos e aumentar também a quantidade de câmeras que cada placa suporta, possivelmente precisando alterar o \textit{hardware} escolhido para um com maior poder de processamento. Outra possibilidade é realizar todo o processamento na nuvem, porém um novo estudo teria que ser feito em relação aos custos desse produto e possíveis mudanças no código.