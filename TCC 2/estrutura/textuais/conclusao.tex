% CONCLUSÃO------------------------------------------------------------
\chapter{CONCLUSÃO}
\label{chap:conclusao}


No desenvolvimento deste trabalho, de forma geral, foram utilizados conhecimentos adquiridos durante todo o curso de engenharia. Utilizaram-se principalmente conhecimentos de eletrônica de potência, eletrônica digital, controle digital e microcontroladores, mas com significativa contribuição de conhecimentos de eletromagnetismo, comunicações digitais e  sensores. O principal desafio do projeto foi a elaboração da fonte inversora controlável, devido à dificuldade de se controlar o magnetron, além das questões de segurança envolvidas, as quais incluem a radiação emitida e a alta tensão presente no secundário.

Quanto ao cumprimento do objetivo geral declarado na introdução, pode-se afirmar que o projeto foi bem sucedido. O objetivo geral foi concluído, visto que o peso e tamanho do circuito de alimentação do magnetron foi reduzido. O consumo de energia também foi reduzido, tendo-se alcançado uma maior eficiência com maior fator de potência. 

Em relação ao objetivos específicos, o circuito inversor controlável foi desenvolvido, sendo o principal componente da solução. O uso do inversor classe E realmente se mostrou um método muito mais robusto e eficaz que a utilização de um transformador ferrorressonante, que é pesado e apresenta grandes perdas por calor em alta potência. A solução de \textit{firmware} desenvolvida  permitiu um controle extremamente rápido e eficiente da potência de saída do magnetron, surpreendendo a equipe com a assertividade do resultado da solução desenvolvida. Na energização dos componentes do circuito, a fonte de alimentação do projeto cumpriu seu papel. Foi possível alimentar o circuito sem qualquer problema, desde o lado primário do transformador até os componentes com tensão DC de dezenas de Volt e os componentes CMOS. A simulação da fonte inversora sem controlador possibilitou a equipe a ter uma boa noção do comportamento do lado primário. Apesar do sucesso das simulações iniciais, as simulações que a equipe almejava realizar não foram concluídas, visto que o lado do secundário é extremamente complexo de se simular, sendo necessário um estudo específico só para levantar a função de transferência do magnetron. A solução conjunta de todos os componentes foi dimensionada de forma correta, podendo ser integrada a um forno microondas convencional, como de fato foi, sendo demonstrado neste trabalho.

No mais, a realização deste trabalho foi extremamente gratificante para equipe. Aprendeu-se muito com o extenso referencial bibliográfico consultado, além de se ter tido a oportunidade de realizar uma solução real para um problema real. Este viés prático e científico é o que compõe a essência o curso de engenharia, exercendo papel fundamental na formação de seus alunos.

Para trabalhos futuros, poderia realizar-se um estudo mais detalhado da função de transferência do magnetron, a fim de utilizar um controlador específico para este dispositivo, através do método do lugar das raízes. Com um controlador especialmente projetado para o controle de potência de magnetron, é bem provável que uma maior eficiência seja alcançada, além de permitir um controle muito mais robusto.