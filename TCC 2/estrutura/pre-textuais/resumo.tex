% RESUMO--------------------------------------------------------------------------------

\begin{resumo}[RESUMO]
\begin{SingleSpacing}

% Não altere esta seção do texto--------------------------------------------------------
\imprimirautorcitacao. \imprimirtitulo. \imprimirdata. \pageref {LastPage} f. \imprimirprojeto\ – \imprimirprograma, \imprimirinstituicao. \imprimirlocal, \imprimirdata.\\
%---------------------------------------------------------------------------------------


Neste projeto, foi desenvolvido um circuito de controle de potência para fornos microondas, que possibilita o controle da potência de saída do magnetron. Este circuito consome menos e energia e tem um tamanho reduzido em relação às fontes ferro-ressonantes tradicionais. O controle de potência é feito através por uma planta que é realimentada por um sensor de temperatura. A planta consiste em uma solução de \textit{firmware} que é integrada a um circuito de controle de alta potência, o qual fará o chaveamento de um inversor ligado ao primário de um transformador de alta potência que possui o magnetron ligado ao secundário. O controle do ciclo de trabalho do chaveamento possibilita o controle de potência de saída do magnetron. A solução de \textit{firmware} juntamente com o circuito de controle consistem em uma unidade com tamanho reduzido, podendo ser integrada à um forno microondas convencional.\\

\textbf{Palavras-chave}: Circuito de controle, microondas, controle de potência.

\end{SingleSpacing}
\end{resumo}

% OBSERVAÇÕES---------------------------------------------------------------------------
% Altere o texto inserindo o Resumo do seu trabalho.
% Escolha de 3 a 5 palavras ou termos que descrevam bem o seu trabalho 
